\documentclass{beamer}
\usepackage{lmodern}
\usepackage[utf8]{inputenc}
\usepackage[english]{babel}

\usepackage{amsmath, amsthm, amsfonts, amssymb}
\usepackage{mathrsfs}

\usetheme{Madrid}
\usecolortheme{default}

\title{Applying the Compactness Theorem}
\author{Alexander Mayorov}
\institute{TU Kaiserslautern}

\def\N{\mathbb{N}_{\ge 0}}
\def\Npos{\mathbb{N}_{\ge 1}}
\def\N{\mathbb{N}_{\ge 0}}
\def\R{\mathbb{R}}
\def\Z{\mathbb{Z}}
\def\Q{\mathbb{Q}}
\def\P{\mathcal{P}}

\begin{document}
	
\frame{\titlepage}

\begin{frame}
	\frametitle{The compactness theorem}
	\begin{theorem}[Compactness for propositional logic]
		A set of propositional formulas $ \Sigma $ is satisfiable if and only if every finite subset of $ \Sigma $ is satisfiable.
	\end{theorem}
	\begin{proof}
		($ \Rightarrow $): If there exists $ \nu \models \Sigma $, then $ \nu $ will satisfy every finite subset of $ \Sigma $.
		
		($ \Leftarrow $): see lecture notes
	\end{proof}

	\pause

	\begin{theorem}[Contrapositive form of compactness]
		$ \Sigma $ is unsatisfiable if and only if there is some finite unsatisfiable subset in $ \Sigma $
	\end{theorem}
\end{frame}

\begin{frame}
	\begin{theorem}[Compactness for first order logic]
		A set of closed first order logic formulas $ \Sigma $ is satisfiable if and only if every finite subset of $ \Sigma $ is satisfiable.
	\end{theorem}
	\pause
	\begin{proof}
		($ \Rightarrow $): A model for $ \Sigma $ is also a model for every finite subset thereof.
		
		($ \Leftarrow $): We can assume without loss of generality that all $ \varphi \in \Sigma $ are Skolem normal form sentences with matrix in CNF (via Tseitin's transformation). Suppose, for a sake of contradiction, $ \Sigma $ were unsatisfiable. Since $ \Sigma $ is satisfiable iff $ E(\Sigma) $ is satisfiable by the Gödel-Herbrand-Skolem theorem, we conclude that $ E(\Sigma) $ is unsatisfiable. By the \alert{compactness theorem for propositional logic}, there must exist some finite unsatisfiable subset $ A \subseteq E(\Sigma) $. Taking formulas from $ \Sigma $ whose ground terms appear in $ A $ gives us a finite unsatisfiable subset of $ \Sigma $.
	\end{proof}
\end{frame}

\begin{frame}
	\frametitle{Strategy for proving non-expressibility results}
	Suppose we want to show that some property \alert{$ P $ is not expressible} in first order logic. Then we can try proving it using the following proof schema:
	\begin{itemize}
		\item<1-> Assume, for a sake of contradiction that there is some formula $ \varphi $ expressing $ P $.
		\item<2-> Construct an infinite set of formulas $ \{\lambda_k \mid k \in \N\} $ that ``approximates'' precisely the negation of $ P $.
		\item<3-> Consider the set $ T = \{\varphi\} \cup \{\lambda_k \mid k \in \N\} $.
		\item<4-> Conclude that $ T $ is unsatisfiable since every model of $ \varphi $ cannot be a model of $ \{\lambda_k \mid k \in \N\} $.
		\item<5-> Show that, on the other hand, for every finite $ M \subseteq T $, there is a model with property $ P $.
		\item<6-> Conclude that, by compactness, $ T $ is satisfiable, meaning that we get a contradiction.
	\end{itemize}
\end{frame}

\begin{frame}
	\frametitle{Finiteness is not first-order definable}
	Finiteness is not first
\end{frame}
	
\end{document}